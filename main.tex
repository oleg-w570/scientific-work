\documentclass[12pt,a4paper]{article}

\usepackage[T1,T2A]{fontenc}
\usepackage[utf8]{inputenc}
\usepackage[russian]{babel}

\usepackage{indentfirst}
\usepackage[a4paper,top=2cm,bottom=2cm,left=2.5cm,right=1.5cm,marginparwidth=1.75cm]{geometry}
\linespread{1.5}

\usepackage{titlesec}
% \titleformat{\section}[hang]{\bfseries}{\thesection.}{1em}{\newpage}
% \titlelabel{\thetitle.\,\,}

\usepackage{amsmath,amsfonts,amssymb}

\usepackage{graphicx}
\usepackage{float}

\usepackage{fontspec}
\setmainfont{Times New Roman}

\begin{document}

\begin{center} % включить выравнивание по центру
    МИНИСТЕРСТВО ОБРАЗОВАНИЯ И НАУКИ РОССИЙСКОЙ \\
    ФЕДЕРАЦИИ \\
    Федеральное государственное автономное образовательное учреждение  \\
    высшего образования\\ \textbf{<<Национальный исследовательский \\ Нижегородский государственный университет \\
    им. Н.И. Лобачевского (ННГУ)>>}\\[1.5cm]%[4.5cm]
    \textbf{Институт информационных технологий, математики и механики}\\[5.5cm]
    %\textbf{Кафедра: Теории управления и динамики систем\\(ТУДС)}\\[0.8cm]
    %Направление подготовки: \textbf{<<Математика и механика>>}\\
    %направленность: \textbf{<<Дифференциальные уравнения, динамические системы и оптимальное управление>>}\\[2.2cm]
    
     \textbf{\large ОТЧЕТ} \\ % название работы, затем отступ 0,6см
     по производственной практике \\[0.6cm]
     
     на тему:\\
      \textbf{\large <<Алгоритмы глобального поиска>>}\\[6.5cm]
     % тема работы, затем отступ 3,7см
    \begin{flushright}
     \begin{minipage}{0.52\textwidth} % начало маленькой врезки в половину ширины текста
     \begin{flushleft} % выровнять её содержимое по левому краю
      \textbf{Выполнили:} \\
     студент группы 382003-3 Зорин О.А. \\
     \end{flushleft} % конец выравнивания по левому краю
     \end{minipage} % конец врезки
    \end{flushright}
     \vfill % заполнить всё доступное ниже пространство
    
      Нижний Новгород \\
     2023
    
     \thispagestyle{empty} % не нумеровать страницу
    
\end{center}

\renewcommand{\contentsname}{Содержание}
\newpage
\tableofcontents

\newpage
\section*{Введение}
Задачи поиска решений возникают перед человеком практически в любой сфере деятельности. Возникающие при этом задачи достаточно редко могут быть решены на основе только интуиции и накопленного опыта исследователя в силу их значительной сложности. Высокая сложность задач обусловлена такими факторами, как большое количество вариантов, нуждающихся в оценке, противоречивость требований, а также изменение самого представления об оптимальности решения в процессе исследования. Поиск решения при выбранных требованиях к оптимальности может быть выполнен с применением алгоритмов глобального поиска.

Возможны различные понятия искомого решения, так как, например, для функции экстремальные свойства можно характеризовать как с помощью локальных экстремумов, так и – глобальных. Как следствие, возникают различные постановки задач оптимизации.

Алгоритмы глобального поиска решают задачи на нахождение глобального минимума функции. Задачи на определение наибольшего значения функции можно свести к задаче минимизации функции, следовательно, алгоритмы глобального поиска применимы и к этим задачам. Также алгоритмы глобального поиска можно использовать при нахождение локального минимума при условии, что функция имеет единственный локальный минимум (в следствие чего, он является глобальным).

Алгоритмы глобального поиска бывают двух видов: детерминированные и недетерминированные. Недетерминированные алгоритмы, как показывает практика, быстрее вычисляют решение и легче реализуемы, но не могут гарантированно утверждать, что найденное решение близко к реальному с заданной точностью. Для гарантированного результата, что обеспечивают доказанные теоремы сходимости, используют детерминированные алгоритмы.  Примером недетерминированных алгоритмов могут служить эволюционные методы, а примером детерминированного алгоритма – алгоритм, предложенный Стронгиным Р.Г.

Данная научно-исследовательской работа проводится с целью изучить, как устроены различные алгоритмы глобального поиска, самостоятельно реализовать один из детерминированных алгоритмов, оценить полученные с помощью него результаты и сравнить их с результатами других алгоритмов.

\section{Постановка задачи}
Пусть имеется множество области поиска, заданное линейными ограничениями
\begin{equation}
    D = \{ y \in \mathbb{R}^N : a_i \le y_i \le b_i, \; 1 \le i \le N \}
\end{equation}
где $y=(y_1, y_2, ..., y_N)$ и $a_i, b_i \in \mathbb{R}$.

На этом множестве определена функция $f(y), y \in D$, которая удовлетворяет условию Липшица с коэффициентом $L$
\begin{equation}
    \exists L>0 \quad \forall y', y'' \in D \quad |f(y') - f(y'')| \le L ||y' - y''||.
\end{equation}

Поставлена задача глобальной оптимизации, т.е. определение глобального минимума данной функции на заданном множестве
\begin{equation}
    \min_{y \in D}{f(y)}.
\end{equation}


\section{Одномерные задачи}
Начнём исследование с рассмотрения простейшего случая – одномерных задач. То есть будем минимизировать функцию на отрезке
\begin{equation}
    f(x) \to \min, \; x \in [a, b] \; (a < b),
\end{equation}
где $a, b \in \mathbb{R}$, а $f(x)$ удовлетворяет условию Липшица
\begin{equation}
    \exists L>0 \quad \forall x', x'' \in [a,b] \quad |f(x') - f(x'')| \le L |x' - x''|.
\end{equation}

\subsection{Метод решения и алгоритмическое описание}
Задача будет решаться с помощью базового алгоритма глобального поиска, предложенным Р.Г. Стронгиным.

Опишем вычислительную схему базового алгоритма глобального поиска. Для поиска оптимального решения в алгоритме глобального поиска выполняются испытания. Под испытанием понимается вычисление значения функции $z=f(x)$. Два первых испытания проводятся на концах отрезка $[a,b]$, т.е. $x^0=a$, $x^1=b$, вычисляются значения функции $z^0=f(a)$, $z^1=f(b)$, и количество проведённых испытаний $k$ становится равным единице. Следующие точки проведения испытаний определяются на основе следующих правил:

















\end{document}